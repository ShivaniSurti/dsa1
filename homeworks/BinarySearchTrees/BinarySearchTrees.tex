\documentclass[paper=a4, fontsize=11pt, parskip=full]{scrartcl} % A4 paper and 11pt font size

\usepackage[T1]{fontenc} % Use 8-bit encoding that has 256 glyphs
\usepackage{fourier} % Use the Adobe Utopia font for the document - comment this line to return to the LaTeX default
\usepackage[english]{babel} % English language/hyphenation
\usepackage{amsmath,amsfonts,amsthm} % Math packages

\usepackage{graphicx}

\usepackage{float}

%Preamble
\usepackage{listings}
\usepackage{color}
\definecolor{javared}{rgb}{0.6,0,0} % for strings
\definecolor{javagreen}{rgb}{0.25,0.5,0.35} % comments
\definecolor{javapurple}{rgb}{0.5,0,0.35} % keywords
\definecolor{javadocblue}{rgb}{0.25,0.35,0.75} % javadoc

\lstset{language=Java,
basicstyle=\ttfamily,
keywordstyle=\color{javapurple}\bfseries,
stringstyle=\color{javared},
commentstyle=\color{javagreen},
morecomment=[s][\color{javadocblue}]{/**}{*/},
numbers=left,
numberstyle=\tiny\color{black},
stepnumber=2,
numbersep=10pt,
tabsize=4,
showspaces=false,
showstringspaces=false}


\usepackage{hyperref}
\hypersetup{
    colorlinks=true,
    linkcolor=blue,
    filecolor=magenta,
    urlcolor=cyan,
}

\usepackage{lipsum} % Used for inserting dummy 'Lorem ipsum' text into the template

\usepackage{sectsty} % Allows customizing section commands
\allsectionsfont{\centering \normalfont\scshape} % Make all sections centered, the default font and small caps

\usepackage{fancyhdr} % Custom headers and footers
\pagestyle{fancyplain} % Makes all pages in the document conform to the custom headers and footers
\fancyhead{} % No page header - if you want one, create it in the same way as the footers below
\fancyfoot[L]{} % Empty left footer
\fancyfoot[C]{} % Empty center footer
\fancyfoot[R]{\thepage} % Page numbering for right footer
\renewcommand{\headrulewidth}{0pt} % Remove header underlines
\renewcommand{\footrulewidth}{0pt} % Remove footer underlines
\setlength{\headheight}{13.6pt} % Customize the height of the header

\numberwithin{equation}{section} % Number equations within sections (i.e. 1.1, 1.2, 2.1, 2.2 instead of 1, 2, 3, 4)
\numberwithin{figure}{section} % Number figures within sections (i.e. 1.1, 1.2, 2.1, 2.2 instead of 1, 2, 3, 4)
\numberwithin{table}{section} % Number tables within sections (i.e. 1.1, 1.2, 2.1, 2.2 instead of 1, 2, 3, 4)

\setlength\parindent{0pt} % Removes all indentation from paragraphs - comment this line for an assignment with lots of text

%----------------------------------------------------------------------------------------
%	TITLE SECTION
%----------------------------------------------------------------------------------------

\newcommand{\horrule}[1]{\rule{\linewidth}{#1}} % Create horizontal rule command with 1 argument of height

\title{
\normalfont \normalsize
\textsc{University of Virginia, Department of Computer Science} \\ [25pt] % Your university, school and/or department name(s)
\horrule{0.5pt} \\[0.4cm] % Thin top horizontal rule
\huge BSTs and AVL - Tree Implementations \\ % The assignment title
\horrule{2pt} \\[0.5cm] % Thick bottom horizontal rule
}

\author{Dr. Mark R. Floryan} % Your name

\date{\normalsize\today} % Today's date or a custom date

\begin{document}

\maketitle % Print the title

%----------------------------------------------------------------------------------------
%	Summary
%----------------------------------------------------------------------------------------

\section{Summary}

For this homework, you will be implementing three classes, which are all increasingly complicated Tree objects. You will start by implementing a basic Binary Tree with the three tree traversals convered in class. Then, you will implement and test a working Binary Search Tree. Lastly, you will implement a few methods to complete an AVL Tree.

\begin{enumerate}
	\item Download the provided starter code
	\item Implement some general methods in the BinaryTree class.
	\item Implement the BinarySearchTree class
	\item Implement the missing methods in the AVLTree class (some of this is done for you to simplify the assignment)
	\item Use the provided tester files to verify your implementation works. Note that you should test your code more so than the provided tester does this time. The tester is NOT as thorough as in previous homeworks.
	\item \textbf{FILES TO DOWNLOAD:} \href{https://markfloryan.github.io/dsa1/homeworks/BinarySearchTrees/code/BinarySearchTrees.zip}{BinarySearchTrees.zip}
	\item \textbf{FILES TO SUBMIT:} TreeImplementations.zip
\end{enumerate}

%------------------------------------------------

\subsection{BinaryTree.java}

To begin, implement the BinaryTree class. These are methods that are useful for ANY binary tree (whether balanced or not). You will need to implement the following methods:

\begin{lstlisting}
public class BinaryTree<T>{
	//Prints the tree using in-order traversal
	private void printInOrder(TreeNode<T> curNode);
	
	//Prints the tree using pre-order traversal
	private void printPreOrder(TreeNode<T> curNode);
	
	//Prints the tree using post-order traversal
	private void printPostOrder(TreeNode<T> curNode);
}
\end{lstlisting}

The Binary Tree contains a few methods that are implemented for you. This includes the main print functions, that simply call the helper functions described above on the root to give off the recursive prints. In addition, a method called printTree() is provided that prints the tree in a somewhat formatted method. This may be useful when debugging your code. Lastly, a simple recursive method that computes the height of the binary tree is provided as an example of recursion in trees.

\subsection{BinarySearchTree.java}

Next, you will implement a binary search tree. This class will extend the binary tree class from earlier, and thus inherit the print methods defined earlier. Your binary search tree should implement the provided Tree interface, shown below. 

\begin{lstlisting}
public interface Tree<T extends Comparable<T>> {
	
	public void insert(T data);
	
	public boolean find(T data);

	public void remove(T data);

	public TreeNode<T> findMax(TreeNode<T> curNode);
}

/* You BST implements the interface above */
public class BinarySearchTree<T extends Comparable<T>>
					extends BinaryTree<T> implements Tree<T>{

		//TODO: Implement this class
}
\end{lstlisting}

You may add other supporting methods to your binary search tree if you find that to be helpful. 

\subsection{AVLTree.java}

Lastly, you will implement an AVL tree that inherits from your binary search tree. An AVL tree can take advantage of the insert and remove methods from the class it inherits from (i.e., BinarySearchTree.java). Thus, to insert into an avl tree, you can call super.insert() and then simply check if the current node needs to be balanced. Some of this implementation is provided for you, but you will have to implement the following methods yourself:

\begin{lstlisting}
public class AVLTree<T extends Comparable<T>>
					extends BinarySearchTree<T>{

	//Insert and remove are partially coded for you already

	//figures out whether a double or single rotation is
	//needed and in which direction(s)
	private TreeNode<T> balance(TreeNode<T> curNode);
			
	//rotate right on the curNode provided
	private TreeNode<T> rotateRight(TreeNode<T> curNode);
	
	//rotate left on the curNode provided
	private TreeNode<T> rotateLeft(TreeNode<T> curNode);
	
	//compute the balance factor of the given node
	private int balanceFactor(TreeNode<T> node);

}
\end{lstlisting}

Once you are done, you can look at the two provided tester files to check your implementation. As stated earlier, these files do not rigorously check your implementations, so you should be writing your own test cases in addition to the few provided.

When you are done, submit your entire project as a zip file.

%------------------------------------------------


%----------------------------------------------------------------------------------------

\end{document}


%----------------------------------------------------------------------------------------
%----------------------------------------------------------------------------------------
%----------------------------------------------------------------------------------------
%----------------------------------------------------------------------------------------
%----------------------------------------------------------------------------------------
%----------------------------------------------------------------------------------------


%WORKS CITED:

%%%%%%%%%%%%%%%%%%%%%%%%%%%%%%%%%%%%%%%%%
% Short Sectioned Assignment
% LaTeX Template
% Version 1.0 (5/5/12)
%
% This template has been downloaded from:
% http://www.LaTeXTemplates.com
%
% Original author:
% Frits Wenneker (http://www.howtotex.com)
%
% License:
% CC BY-NC-SA 3.0 (http://creativecommons.org/licenses/by-nc-sa/3.0/)
%
%%%%%%%%%%%%%%%%%%%%%%%%%%%%%%%%%%%%%%%%%

%----------------------------------------------------------------------------------------
%	PACKAGES AND OTHER DOCUMENT CONFIGURATIONS
%----------------------------------------------------------------------------------------
